\documentclass{article}
\title{ Titolo del documento}
\date{}
\author{}
\begin{document}
\maketitle
\section{ Sezione 1}
 Questo è un paragrafo. Corto ma pur sempre un paragrafo.

 Anche questo è un paragrafo

 Che ore sono?

\section{ Sezione 2}
 Un paragrafo ancora.

\subsection{ Sezione 2.1}
 I paragrafi possono apparire dentro le sottosezioni.

 Altro paragrafo.

\section{ Principi fondamentali della Costituzione Italiana}
\subsection{ Art. 1}
 L'Italia è una Repubblica democratica, fondata sul lavoro.

 La sovranità appartiene al popolo, che la esercita nelle forme e nei limiti della Costituzione.

\subsection{ Art. 2}
 La Repubblica riconosce e garantisce i diritti inviolabili dell'uomo, sia come singolo sia nelle formazioni sociali ove si svolge la sua personalità, e richiede l'adempimento dei doveri inderogabili di solidarietà politica, economica e sociale.

\subsection{ Art. 3}
 Tutti i cittadini hanno pari dignità sociale e sono eguali davanti alla legge, senza distinzione di sesso, di razza, di lingua, di religione, di opinioni politiche, di condizioni personali e sociali.

 È compito della Repubblica rimuovere gli ostacoli di ordine economico e sociale, che, limitando di fatto la libertà e l'eguaglianza dei cittadini, impediscono il pieno sviluppo della persona umana e l'effettiva partecipazione di tutti i lavoratori all'organizzazione politica, economica e sociale del Paese.

\subsection{ Art. 4}
 La Repubblica riconosce a tutti i cittadini il diritto al lavoro e promuove le condizioni che rendano effettivo questo diritto.

 Ogni cittadino ha il dovere di svolgere, secondo le proprie possibilità e la propria scelta, un'attività o una funzione che concorra al progresso materiale o spirituale della società.

\subsection{ Art. 5}
 La Repubblica, una e indivisibile, riconosce e promuove le autonomie locali; attua nei servizi che dipendono dallo Stato il più ampio decentramento amministrativo; adegua i principi ed i metodi della sua legislazione alle esigenze dell'autonomia e del decentramento.

\subsection{ Art. 6}
 La Repubblica tutela con apposite norme le minoranze linguistiche.

\subsection{ Art.7}
 Lo Stato e la Chiesa cattolica sono, ciascuno nel proprio ordine, indipendenti e sovrani.

 I loro rapporti sono regolati dai Patti Lateranensi. Le modificazioni dei Patti accettate dalle due parti, non richiedono procedimento di revisione costituzionale.

\subsection{ Art. 8}
 Tutte le confessioni religiose sono egualmente libere davanti alla legge.

 Le confessioni religiose diverse dalla cattolica hanno diritto di organizzarsi secondo i propri statuti, in quanto non contrastino con l'ordinamento giuridico italiano.

 I loro rapporti con lo Stato sono regolati per legge sulla base di intese con le relative rappresentanze.

\subsection{ Art. 9}
 La Repubblica promuove lo sviluppo della cultura e la ricerca scientifica e tecnica.

 Tutela il paesaggio e il patrimonio storico e artistico della Nazione.

\subsection{ Art. 10}
 L'ordinamento giuridico italiano si conforma alle norme del diritto internazionale generalmente riconosciute.

 La condizione giuridica dello straniero è regolata dalla legge in conformità delle norme e dei trattati internazionali.

 Lo straniero, al quale sia impedito nel suo paese l'effettivo esercizio delle libertà democratiche garantite dalla Costituzione italiana, ha diritto d'asilo nel territorio della Repubblica secondo le condizioni stabilite dalla legge.

 Non è ammessa l'estradizione dello straniero per reati politici.

\subsection{ Art. 11}
 L'Italia ripudia la guerra come strumento di offesa alla libertà degli altri popoli e come mezzo di risoluzione delle controversie internazionali; consente, in condizioni di parità con gli altri Stati, alle limitazioni di sovranità necessarie ad un ordinamento che assicuri la pace e la giustizia fra le Nazioni; promuove e favorisce le organizzazioni internazionali rivolte a tale scopo.

\subsection{ Art. 12}
 La bandiera della Repubblica è il tricolore italiano: verde, bianco e rosso, a tre bande verticali di eguali dimensioni.

\end{document}
